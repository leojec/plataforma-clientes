\documentclass{article}
\usepackage[utf8]{inputenc}
\usepackage{graphicx}
\usepackage{geometry}
\usepackage{enumitem}
\usepackage{titlesec}
\usepackage{hyperref}
\usepackage{float}
\geometry{a4paper, margin=2.5cm}


\begin{document}

\begin{titlepage}
    \centering
    \includegraphics[width=0.3\textwidth]{tcece/images/catolica.png}\par\vspace{1cm}
    {\Large Centro Universitário Católica de Santa Catarina - Joinville\par}
    {\large Engenharia de Software\par}
    \vspace{2cm}
    {\large Leonardo Luis da Rocha\par}
    \vspace{6cm}
    {\Large \textbf{CRMSHOT - CRM De Vendas}}\par
    {\large de feiras e eventos\par}
    \vspace{11cm}
    {\large Novembro de 2025\par}
\end{titlepage}



\newpage

\section*{Resumo}
\addcontentsline{toc}{section}{Resumo}

O projeto consiste no desenvolvimento de um sistema de CRM (Customer Relationship Management) para o evento Shot Fair Brasil que permite o cadastro e o acompanhamento eficiente de clientes ao longo de todo o processo comercial. A plataforma foi desenvolvida e implementada para otimizar a gestão do relacionamento com os clientes, garantindo um fluxo de trabalho organizado e transparente para a equipe de vendas. O sistema inclui funcionalidades de pipeline de vendas visual (Kanban), dashboard analítico, agenda de atividades, relatórios gerenciais e um assistente CRM inteligente.

\section{Introdução}

\subsection{Contexto}

A gestão de relacionamento com clientes é um fator crítico para o sucesso comercial de empresas em mercados competitivos. No entanto, muitas organizações ainda utilizam planilhas ou anotações manuais, resultando em perda de dados, dificuldades na comunicação interna e baixa eficiência no acompanhamento de clientes. A adoção de um sistema CRM visa resolver esses entraves por meio da automatização e centralização dos processos de vendas e atendimento ao cliente.

\subsection{Justificativa}

A implementação de um sistema de CRM é essencial para promover uma gestão estruturada e inteligente do relacionamento com os clientes. Por meio dele, a equipe de vendas poderá acompanhar cada estágio do funil de vendas, realizar interações com registro histórico e obter indicadores de desempenho em tempo real. Diferente de soluções genéricas ou manuais, este projeto foca em uma proposta customizável, segura e com foco específico na melhoria da performance comercial.

\subsection{Objetivos}

\textbf{Objetivo principal:} Desenvolver um sistema web de CRM que automatize o cadastro, acompanhamento e análise do relacionamento com clientes, desde o primeiro contato até o fechamento da venda.

\vspace{0.5cm}

\textbf{Objetivos específicos:}

\vspace{-0.2cm}
\begin{itemize}
    \item Levantar os requisitos funcionais e não funcionais do sistema.
    \item Especificar os módulos principais: cadastro, pipeline de vendas, histórico e relatórios.
    \item Modelar a base de dados e definir os fluxos de interação da aplicação.
    \item Desenvolver e integrar os componentes frontend e backend.
    \item Testar e validar a usabilidade e a segurança do sistema.
\end{itemize}

\vspace{0.5cm}

\textbf{Status do Projeto:} O sistema foi completamente desenvolvido e implementado, com backend e frontend funcionais, banco de dados estruturado, API REST completa, interface responsiva, autenticação JWT implementada e cobertura de testes unitários e de integração.

\newpage

\section{Descrição do Projeto}

\subsection{Tema do Projeto}

O projeto tem como tema o desenvolvimento de um sistema de CRM (Customer Relationship Management) para a empresa que buscam otimizar o acompanhamento e a gestão do relacionamento com clientes e prospects. O sistema será projetado como uma plataforma web acessível por navegadores modernos, com interface intuitiva, adaptada tanto para desktops quanto para dispositivos móveis.

\subsection{Escopo Funcional}

O sistema CRM implementado permite:
\begin{itemize}[leftmargin=*, label=--]
    \item O cadastro e gestão completa de expositores (CRUD), incluindo dados corporativos, validação de CNPJ em tempo real, busca e filtros avançados, e histórico completo de interações.
    \item O acompanhamento do pipeline de vendas através de visualização Kanban com drag and drop, permitindo mover oportunidades entre estágios (Lead, Em Andamento, Em Negociação, Stand Fechado) com atualização em tempo real.
    \item O registro detalhado de interações entre o time de vendas e os expositores (ligações, e-mails, reuniões, visitas), permitindo rastreabilidade completa com histórico por expositor.
    \item A configuração de lembretes, tarefas e follow-ups através de agenda de atividades com calendário de interações, marcação de conclusão e lembretes automáticos.
    \item A geração de relatórios gerenciais e dashboards analíticos com estatísticas gerais, gráficos interativos, métricas de vendas (metros quadrados, propostas, ganhos) e visualização por período e por vendedor.
    \item O controle de usuários com diferentes perfis (administrador, gerente, vendedor), incluindo autenticação segura com JWT e criptografia de senhas com BCrypt.
    \item Um assistente CRM inteligente (chatbot) para consultas rápidas sobre próximas reuniões, quantidade de leads, atividades do dia, com respostas em tempo real baseadas nos dados do sistema.
\end{itemize}

\subsection{Problemas a Resolver}

\begin{itemize}[leftmargin=*, label=--]
    \item Falta de centralização das informações, que normalmente ficam dispersas em planilhas individuais ou sistemas genéricos, dificultando a visão completa do cliente.
    \item Alto risco de perda de oportunidades por ausência de lembretes e falta de rastreabilidade no relacionamento com o cliente.
    \item Dificuldade para gerentes visualizarem a performance da equipe em tempo real, comprometendo a tomada de decisões estratégicas.
    \item Dificuldade de consolidar dados históricos para entender padrões de comportamento dos expositores e oportunidades para negócios futuros.
\end{itemize}

\newpage

\subsection{Diferenciais do Projeto}

Este CRM foi desenvolvido com foco específico no segmento de eventos e feiras, implementando:
\begin{itemize}[leftmargin=*, label=--]
    \item Pipeline visual Kanban com drag and drop para gestão intuitiva de oportunidades de vendas, permitindo movimentação entre estágios de forma interativa.
    \item Dashboard analítico com gráficos interativos (CanvasJS e Recharts) para visualização de métricas de performance, vendas e atividades em tempo real.
    \item Assistente CRM inteligente integrado que responde consultas sobre o sistema, facilitando o acesso rápido a informações importantes.
    \item Interface responsiva desenvolvida com Tailwind CSS, adaptada para desktops e dispositivos móveis, garantindo usabilidade em diferentes plataformas.
    \item Integração com API da Receita Federal para validação automática de CNPJ durante o cadastro de expositores.
    \item Arquitetura preparada para deploy em nuvem (AWS) com suporte a RDS PostgreSQL, Elastic Beanstalk e CloudFront.
\end{itemize}

\subsection{Limitações}

Apesar do foco robusto na gestão comercial, a primeira versão do sistema implementada possui algumas limitações estratégicas:
\begin{itemize}[leftmargin=*, label=--]
    \item Não inclui integrações diretas com gateways de pagamento ou ERPs legados, sendo estas funcionalidades planejadas para futuras releases.
    \item Não contempla automação de campanhas de marketing digital, deixando este módulo para futuras iterações.
    \item A exportação de relatórios em PDF e Excel está preparada mas ainda não totalmente implementada na interface, sendo uma funcionalidade em desenvolvimento.
    \item A cobertura de segurança atual é baseada em autenticação JWT e criptografia BCrypt, com monitoramento avançado e autenticação de dois fatores (MFA) planejados para iterações futuras.
\end{itemize}

\section{Especificação Técnica}

\subsection{Modelagem C4}

\begin{minipage}{\textwidth}
    A modelagem C4 é uma abordagem para descrever a arquitetura de software por meio de uma hierarquia de diagramas que vai do geral para o detalhado. 
    O método foi proposto por Simon Brown e organiza a visualização em quatro níveis: \textbf{Contexto}, \textbf{Contêiner}, \textbf{Componente} e \textbf{Código}, 
    permitindo comunicar a arquitetura tanto para stakeholders não técnicos quanto para desenvolvedores.

    \vspace{0.5cm}
    \centering
    \includegraphics[width=0.95\textwidth]{images/tccimg.drawio.png}
\end{minipage}





\subsection{Diagrama de Caso de Uso}

\subsubsection{Vendedor}

\begin{minipage}{\textwidth}
\vspace{0.3cm}
O vendedor é o principal usuário operacional do sistema CRM. Suas responsabilidades envolvem cadastrar e gerenciar expositores, controlar o pipeline de vendas, registrar históricos de interações, configurar lembretes automáticos e gerar relatórios. Além disso, pode exportar dados em PDF/Excel, gerenciar calendários de follow-ups e registrar feedbacks dos clientes para futuras abordagens.

\vspace{0.5cm}
\centering
\includegraphics[width=0.6\textwidth]{images/vendedorcuso.png}
\end{minipage}


\subsubsection{Administrador}

O administrador é responsável pela gestão avançada do sistema, incluindo o gerenciamento de usuários e permissões, auditoria e visualização de relatórios gerenciais e dashboards consolidados. Dessa forma, ele possui uma visão estratégica do CRM, possibilitando monitorar o desempenho da equipe e tomar decisões baseadas em dados.

\begin{figure}[H]
    \centering
    \includegraphics[width=0.6\textwidth]{images/admcuso.png}
    \label{fig:caso-uso-admin}
\end{figure}

\newpage

\subsection{Requisitos Funcionais}

\begin{itemize}[leftmargin=*, label=--]
    \item \textbf{RF01:} O sistema permite que o \textbf{vendedor} cadastre expositores com dados como razão social, nome fantasia, CNPJ (com validação em tempo real), e-mail, telefone, endereço completo e descrição. \textbf{Status: Implementado}
    \item \textbf{RF02:} O sistema permite que o \textbf{vendedor} gerencie oportunidades de vendas através de pipeline visual Kanban com drag and drop, permitindo atualização de status, valores estimados, probabilidade de fechamento e data prevista de fechamento. \textbf{Status: Implementado}
    \item \textbf{RF03:} O sistema registra o histórico de interações do \textbf{vendedor} com expositores, incluindo chamadas, reuniões, e-mails e visitas, com data, hora, tipo de interação, assunto, descrição e próxima ação. \textbf{Status: Implementado}
    \item \textbf{RF04:} O sistema permite que o \textbf{vendedor e administrador} visualizem relatórios e dashboards com indicadores de desempenho, incluindo estatísticas gerais, gráficos interativos de atividades, métricas de vendas (metros quadrados, propostas, ganhos) e visualização por período. \textbf{Status: Implementado}
    \item \textbf{RF05:} O sistema permite que o \textbf{administrador} gerencie usuários com controle de acesso baseado em perfis (Admin, Vendedor, Gerente) através de autenticação JWT e Spring Security. \textbf{Status: Implementado}
    \item \textbf{RF06:} O sistema permite que o \textbf{vendedor} configure lembretes e tarefas através da agenda de atividades, com marcação de conclusão e histórico completo de atividades por expositor. \textbf{Status: Implementado}
    \item \textbf{RF07:} O sistema permite que o \textbf{vendedor} gerencie informações de oportunidades associadas a expositores, incluindo título, descrição, valor estimado, fonte da oportunidade e status no pipeline. \textbf{Status: Implementado}
    \item \textbf{RF08:} O sistema permite que o \textbf{vendedor} registre informações detalhadas nas interações, incluindo valor de proposta e metros quadrados, que podem ser utilizados para análise de preferências dos expositores. \textbf{Status: Implementado}
    \item \textbf{RF09:} O sistema permite que o \textbf{gerente} visualize dashboards consolidados com métricas de equipe, performance individual através de relatórios de performance de vendedores e análise de oportunidades por status. \textbf{Status: Implementado}
    \item \textbf{RF10:} O sistema está preparado para exportação de relatórios em formatos PDF e Excel, com endpoints de API prontos, sendo a interface de exportação uma funcionalidade em desenvolvimento. \textbf{Status: Parcialmente Implementado}
    \item \textbf{RF11:} O sistema permite que o \textbf{vendedor} gerencie um calendário de atividades (agenda) com data de próxima ação, permitindo acompanhamento de follow-ups e marcação de conclusão de tarefas. \textbf{Status: Implementado}
    \item \textbf{RF12:} O sistema permite que o \textbf{vendedor} registre feedback e observações detalhadas sobre cada expositor e interação através do campo descrição nas interações, melhorando futuras abordagens comerciais. \textbf{Status: Implementado}
    \item \textbf{RF13:} O sistema inclui um assistente CRM inteligente (chatbot) que permite consultas rápidas sobre o sistema, como próximas reuniões, quantidade de leads e atividades do dia. \textbf{Status: Implementado}
\end{itemize}

\newpage

\subsection{Requisitos Não Funcionais}

\begin{itemize}[leftmargin=*, label=--]
    \item \textbf{RNF01:} O sistema utiliza autenticação segura com JWT (jjwt 0.11.5) e criptografia de senhas com BCrypt através do Spring Security, com expiração de tokens configurável. \textbf{Status: Implementado}
    \item \textbf{RNF02:} O sistema foi desenvolvido com arquitetura modular baseada em API REST, permitindo deploy independente de frontend e backend. A infraestrutura está preparada para escalabilidade horizontal através de AWS Elastic Beanstalk e RDS. \textbf{Status: Implementado}
    \item \textbf{RNF03:} O sistema apresenta design responsivo desenvolvido com Tailwind CSS, adaptado para dispositivos desktop e mobile, seguindo padrões de usabilidade modernos. \textbf{Status: Implementado}
    \item \textbf{RNF04:} O sistema foi otimizado para responder a requisições de forma eficiente, com uso de React Query para cache de requisições no frontend e otimizações de queries no backend. \textbf{Status: Implementado}
    \item \textbf{RNF05:} O sistema está preparado para alta disponibilidade através de deploy em AWS (Elastic Beanstalk, RDS com backups automáticos), com monitoramento de performance configurável. \textbf{Status: Implementado}
    \item \textbf{RNF06:} O sistema suporta múltiplos usuários simultâneos através de arquitetura stateless com JWT e conexões de banco de dados gerenciadas pelo Spring Data JPA. \textbf{Status: Implementado}
    \item \textbf{RNF07:} O sistema está configurado para backup automático através de AWS RDS, que oferece backups automáticos configuráveis com retenção personalizável. \textbf{Status: Implementado}
    \item \textbf{RNF08:} O sistema é compatível com navegadores modernos (Chrome, Firefox, Safari, Edge) através do uso de React 18.2 e padrões web modernos. \textbf{Status: Implementado}
    \item \textbf{RNF09:} O sistema implementa testes unitários e de integração com JUnit 5 e Mockito, com cobertura de código medida através de JaCoCo, garantindo qualidade e manutenibilidade. \textbf{Status: Implementado}
    \item \textbf{RNF10:} O sistema implementa validação de dados tanto no frontend (React Hook Form) quanto no backend (Bean Validation), garantindo integridade dos dados. \textbf{Status: Implementado}

\end{itemize}

\subsection{Tecnologias Utilizadas}

\begin{itemize}[leftmargin=*, label=--]
    \item \textbf{Frontend:} React 18.2, Tailwind CSS 3.2.7, React Router DOM 6.8.1, Axios 1.3.4, React Query 3.39.3, @dnd-kit (drag and drop), CanvasJS 1.8.3, Recharts 2.5.0, Lucide React Icons
    \item \textbf{Backend:} Spring Boot 3.2.0, Java 17, Spring Data JPA/Hibernate, Spring Security, Maven
    \item \textbf{Banco de Dados:} PostgreSQL (local e AWS RDS)
    \item \textbf{Autenticação:} JWT (jjwt 0.11.5), BCrypt para criptografia de senhas
    \item \textbf{Testes:} JUnit 5, Mockito, JaCoCo (cobertura de testes)
    \item \textbf{Infraestrutura:} AWS Elastic Beanstalk (backend), AWS RDS PostgreSQL, AWS S3 + CloudFront (frontend)
\end{itemize}

\newpage

\subsection{Justificativa das Tecnologias}

As tecnologias foram escolhidas visando atender requisitos de desempenho, escalabilidade, segurança e facilidade de manutenção do projeto:

\begin{itemize}[leftmargin=*, label=--]
    \item \textbf{React 18.2}: Framework amplamente utilizado para o desenvolvimento frontend, oferece alto desempenho na renderização por meio do Virtual DOM, rica comunidade e ecosistema de componentes prontos. A integração com Tailwind CSS facilita a criação de interfaces responsivas e modernas, enquanto React Query otimiza o gerenciamento de estado e cache de requisições.
    \item \textbf{Spring Boot 3.2.0 (Java 17)}: Escolhido para o backend por proporcionar rapidez no desenvolvimento de aplicações RESTful, robustez e extensões para segurança via Spring Security. A integração com JPA/Hibernate facilita o mapeamento objeto-relacional e a compatibilidade com PostgreSQL, enquanto o Maven gerencia dependências e build do projeto.
    \item \textbf{PostgreSQL}: Banco de dados relacional maduro e confiável, com forte suporte a integridade referencial, transações ACID, extensões como JSONB para dados semiestruturados e recursos avançados para otimização de consultas. A compatibilidade com AWS RDS permite escalabilidade e alta disponibilidade em produção.
    \item \textbf{JWT (jjwt 0.11.5)}: Token JWT é um padrão consolidado para autenticação stateless, permitindo fácil integração com APIs e escalabilidade horizontal, já que a validação do token independe de sessão armazenada em servidor. A criptografia de senhas com BCrypt garante segurança adicional.
    \item \textbf{Tailwind CSS}: Framework CSS utility-first que acelera o desenvolvimento de interfaces responsivas e modernas, permitindo estilização rápida sem necessidade de CSS customizado extensivo.
    \item \textbf{@dnd-kit}: Biblioteca moderna para implementação de drag and drop no pipeline Kanban, oferecendo melhor performance e acessibilidade comparado a soluções antigas.
    \item \textbf{CanvasJS e Recharts}: Bibliotecas de gráficos que permitem visualização interativa de dados no dashboard, facilitando análise de métricas e performance.
\end{itemize}

\newpage
\section{Status de Implementação e Próximos Passos}

\subsection*{Funcionalidades Implementadas}

O sistema foi completamente desenvolvido e implementado, incluindo:

\begin{itemize}[leftmargin=*, label=--]
    \item \textbf{Backend:} API REST completa com Spring Boot 3.2.0, incluindo módulos de autenticação, expositores, oportunidades, interações, agenda, dashboard, relatórios e assistente CRM.
    \item \textbf{Frontend:} Interface completa desenvolvida com React 18.2, incluindo páginas de login, dashboard, pipeline Kanban, gestão de expositores, agenda, relatórios e chatbot.
    \item \textbf{Banco de Dados:} Estrutura completa com entidades relacionadas (Usuario, Expositor, Oportunidade, Interacao) e relacionamentos configurados.
    \item \textbf{Autenticação:} Sistema de autenticação JWT implementado com Spring Security e criptografia BCrypt.
    \item \textbf{Testes:} Cobertura de testes unitários e de integração com JUnit 5, Mockito e JaCoCo.
    \item \textbf{Deploy:} Configuração para deploy em AWS (Elastic Beanstalk, RDS, S3, CloudFront).
\end{itemize}

\subsection*{Melhorias e Funcionalidades Futuras}

\begin{itemize}[leftmargin=*, label=--]
    \item \textbf{Exportação de Relatórios:} Finalizar implementação da exportação de relatórios em PDF e Excel na interface do usuário.
    \item \textbf{Performance:} Implementar cache Redis para consultas frequentes e otimização adicional de queries do banco de dados.
    \item \textbf{Notificações:} Implementar notificações push para lembretes e follow-ups automáticos.
    \item \textbf{Integrações:} Desenvolver integrações com calendários externos (Google Calendar, Outlook) e sistemas de pagamento.
    \item \textbf{Segurança Avançada:} Implementar autenticação de dois fatores (2FA), rate limiting nas APIs e auditoria completa de ações.
    \item \textbf{Mobile:} Desenvolver aplicativo mobile com React Native para acesso offline e notificações mobile.
    \item \textbf{Dashboard Customizável:} Permitir que usuários personalizem seus dashboards com widgets configuráveis.
\end{itemize}


\section{Referências}

\begin{itemize}[leftmargin=*, label=--]
    \item \texttt{https://spring.io/projects/spring-boot}
    \item \texttt{https://www.postgresql.org/docs/}
    \item \texttt{https://reactjs.org}
    \item \texttt{https://tailwindcss.com}
    \item \texttt{https://tanstack.com/query}
    \item \texttt{https://jwt.io/introduction}
    \item \texttt{https://owasp.org/www-project-top-ten/}
    \item \texttt{https://github.com/kelektiv/node.bcrypt.js}
    \item \texttt{https://docs.dndkit.com}
    \item \texttt{https://canvasjs.com}
    \item \texttt{https://recharts.org}
    \item Fielding, Roy T. \textit{Architectural Styles and the Design of Network-based Software Architectures}, 2000.
    \item Martin, Robert C. \textit{Clean Architecture}, 2017.
    \item AWS Documentation. \textit{Amazon RDS User Guide}. Disponível em: \texttt{https://docs.aws.amazon.com/rds}
    \item AWS Documentation. \textit{Elastic Beanstalk Developer Guide}. Disponível em: \texttt{https://docs.aws.amazon.com/elasticbeanstalk}
\end{itemize}

\section{Avaliações de Professores}

\begin{flushleft}
Assinatura Prof Edicarsia:

\vspace{1cm}

Assinatura Prof Camargo:

\vspace{1cm}

Assinatura Prof Manseira:
\end{flushleft}


\end{document}

